\documentclass[12pt]{report}


% Title Page
\title{
{COLLEGE OF COMPUTING AND INFORMATION SCIENCES}\\
{\large BACHELOR OF SCIENCE IN COMPUTER SCIENCE}\\
{\Large \textbf{A CONCEPT PAPER ON ASSESSING THE PERFORMANCE OF STOCHASTIC METHODS IN STRUCTURAL MODELLING}}\\
}
\author{
{KABUUNGA HAMIDU 15/U/5389/PS}\\
{MUSINGUZI WILLIAM 14/U/10056/PS}\\
{WAMALA EMMANUEL 15/U/13659/PS}\\
{BADAZA PROTUS WAISWA 14/U/3832/EVE}
\textbf{}}


\begin{document}
\maketitle
\section{Introduction}
{Stochastic structural optimization avoids highly specialized designs and there-
	fore reduces imperfection sensitivity. It naturally includes statistical uncertain-
	ties into the design optimization process. Furthermore, it allows the inclusion of
	quality control measures (manufacturing, maintenance) into the design process.
	It is, however, computationally very expensive unless based on approximations
	such as response surface models.}
\section{Background to the problem}
{In many engineering application there is an increasing demand on the availabil-
	ity of tools to incorporate unavoidable random variability of loads and system
	properties into the work- ow of structural analysis. This requires a close relation
	between the data structures as required for traditional Finite Element analyses
	and the stochastic tool to obtain a suitable statistical description of the relevant
	responses. This is readily achievable by using established software development
	environments such as e.g. C++. Due to the required compilation process and
	the possibly code optimization associated with it, the computational perfor-
	mance can be quite impressive. On the other hand, the compile-link-cycles do
	not allow for quick checks how minor algorithmic modi
	cations or extensions af- fect the quality of desired results. This is particu-
	larly annoying when developing larger software projects in a distributed system.}
\section{Problem statement}
{Due to the ever increasing demand on performance and cost-ecacy of struc-
	tures, the need for numerical tools to optimize such structures in the design
	process has become very strong. The computational demand arising from op-
	timization methods is quite heavy, and it is even more increasing since various
	stochastic uncertainties have to be taken into account in the design optimization
	process, because of this we came out with an aid for accessing the performance
	of stochastic optimization methods in structural modeling.}
\section{Objectives}
{1. Main Objective To access the performance of stochastic optimization methods
	in structural mod- eling.\\ 2.The other Objective to collect data from the various
	sources about stochastic optimization meth- ods in structural modeling. To
	study and analyze the current exam entry veri
	cation system and data col- lected with a clear aim of identifying weaknesses,
	strengths and requirements of the new system. To design and implement a pro-
	totype of access the performance of stochastic optimization methods in struc-
	1
	tural modeling. To test and validate the performance of stochastic optimization
	methods in structural modeling to ensure that it is error free .}
\section{Methodology}
{We intend to use observation, interviews, and review literation of the collected
	data from the various sources about stochastic optimization methods in struc-
	tural modelling. The data collected will be subjected to analysis in order to
	ensure consistency in modelling the system. The data collected will be grouped
	into functional and non-functional requirements. Functional requirements will
	specify what the system will do while the non-functional requirements will be
	the overall constraints. Tools such as SPSS are to be used as they are auto-
	mated and give quick results. The system will be designed by the use of use
	case diagrams and data ow diagrams that will help to explain how the ac-
	tors will interact with the system. We intend to use visual studio and C++
	to implement and access the performance of stochastic optimization methods
	in structural modeling in order to produce a prototype basing on the user re-
	quirements. The system will be tested to achieve the forth objective using unit
	testing and system testing. Thereafter the system will be taken to the users for
	validation to ensure that it operates to their satisfaction.}
\section{Outcomes}
{The proposed system is to access the performance of stochastic optimization
	methods in structural modelling since the architecture is intended to ease the
	accessing process Furthermore, the proposed system whose research is being
	carried out will be very useful to any other researchers who may wish to venture
	into still the same or nearly the same developments in the technology world.}
\section{References}
\begin{thebibliography}{1}
\bibitem{}{Arsham, H. (1998), “Techniques for Monte Carlo Optimizing,” Monte Carlo Methods and Applications, vol. 4, pp. 181−229.\\
\bibitem{}{	Blum, J. R. (1954), “Multidimensional Stochastic Approximation Methods,” Annals of Mathematical Statistics, vol. 25, pp. 737−744.} \\
\bibitem{}{Sakalauskas, L. (1997). A centering by the Monte-Carlo method. Stochastic Analysis and Applications, 4(15).\\}
\bibitem{}	{Shapiro, A. (1989). Asymptotic properties of statistical estimators in stochastic programming. The Annals of Statistics, 2(17), 841–858.}
\end{thebibliography}
		
		

\begin{abstract}
\end{abstract}

\end{document}          

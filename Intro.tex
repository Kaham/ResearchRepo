\documentclass[12pt]{report}


% Title Page
\title{
{COLLEGE OF COMPUTING AND INFORMATION SCIENCES}\\
{\large BACHELOR OF SCIENCE IN COMPUTER SCIENCE}\\
{\Large \textbf{A CONCEPT PAPER ON ASSESSING THE PERFORMANCE OF STOCHASTIC METHODS IN STRUCTURAL MODELLING}}\\
}
\author{
{KABUUNGA HAMIDU 15/U/5389/PS}\\
{MUSINGUZI WILLIAM 14/U/10056/PS}\\
{WAMALA EMMANUEL 15/U/13659/PS}\\
{BADAZA PROTUS WAISWA 14/U/3832/EVE}
\textbf{}}


\begin{document}
\maketitle
\section{Introduction}
{Stochastic structural optimization avoids highly specialized designs and there-
	fore reduces imperfection sensitivity. It naturally includes statistical uncertain-
	ties into the design optimization process. Furthermore, it allows the inclusion of
	quality control measures (manufacturing, maintenance) into the design process.
	It is, however, computationally very expensive unless based on approximations
	such as response surface models.}
\section{Background to the problem}
{In many engineering application there is an increasing demand on the availabil-
	ity of tools to incorporate unavoidable random variability of loads and system
	properties into the work- ow of structural analysis. This requires a close relation
	between the data structures as required for traditional Finite Element analyses
	and the stochastic tool to obtain a suitable statistical description of the relevant
	responses. This is readily achievable by using established software development
	environments such as e.g. C++. Due to the required compilation process and
	the possibly code optimization associated with it, the computational perfor-
	mance can be quite impressive. On the other hand, the compile-link-cycles do
	not allow for quick checks how minor algorithmic modi
	cations or extensions af- fect the quality of desired results. This is particu-
	larly annoying when developing larger software projects in a distributed system.}
\

\begin{abstract}
\end{abstract}

\end{document}          
